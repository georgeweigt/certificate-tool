\documentclass[12pt]{article}
\usepackage{amsmath}
\usepackage[margin=2cm]{geometry}
\begin{document}

\section{Elliptic curve functions}

\bigskip
\bigskip
\begin{verbatim}
void ec256_encrypt
          (struct keyinfo *key, uint8_t *hash, int len, uint8_t *sig)
\end{verbatim}

\noindent
Encrypts {\tt hash} using the cipher prime256v1.
The 64 byte result is returned in {\tt sig}.

\bigskip
\begin{tabular}{ll}
{\tt key} & Pointer to a private key.\\
{\tt hash} & Pointer to a hash digest value.\\
{\tt len} & Byte length of hash digest value (32 bytes maximum).\\
{\tt sig} & Pointer to a 64 byte buffer.
\end{tabular}

\bigskip
\bigskip
\begin{verbatim}
void ec384_encrypt
          (struct keyinfo *key, uint8_t *hash, int len, uint8_t *sig)
\end{verbatim}

\noindent
Encrypts {\tt hash} using the cipher secp384r1.
The 96 byte result is returned in {\tt sig}.

\bigskip
\begin{tabular}{ll}
{\tt key} & Pointer to a private key.\\
{\tt hash} & Pointer to a hash digest value.\\
{\tt len} & Byte length of hash digest value (48 bytes maximum).\\
{\tt sig} & Pointer to a 96 byte buffer.
\end{tabular}

\bigskip
\bigskip
\begin{verbatim}
int ec256_verify
          (struct certinfo *p, struct certinfo *q)
\end{verbatim}

\noindent
Returns 0 if certificate {\tt p} is signed by {\tt q}
where the public key type of {\tt q} is prime256v1.

\bigskip
\bigskip
\begin{verbatim}
int ec384_verify
          (struct certinfo *p, struct certinfo *q)
\end{verbatim}

\noindent
Returns 0 if certificate {\tt p} is signed by {\tt q}
where the public key type of {\tt q} is secp384r1.

\newpage
\section{Signing algorithm}

\noindent
Consider two certificates $P$ and $Q$ where $P$ is signed by $Q$.
Let $s$ be the signature in $P$ and let $k$ be the public
key in $Q$.

\begin{center}
\begin{tabular}{|c|}
\hline
$\quad P\quad$\\
\\
\\
\\
\\
\\
\hline
$s$\\
\hline
\end{tabular}
\qquad
\begin{tabular}{|c|}
\hline
$\quad Q\quad$\\
\\
\\
\hline
$k$\\
\hline
\\
\\
\\
\hline
\end{tabular}
\end{center}

\noindent
By necessity $s$ and $k$ are compatible.
For example, if $k$ is RSA 2048 then $s$ is a PKCS signature that is 2048 bits in length
(256 bytes).

\bigskip
\noindent
A hash digest of $P$ is contained in $s$.
For example, the following unencrypted signature $s$ is for RSA 2048 and hash digest SHA256.
(Numerals are in hexadecimal.)

\begin{center}
\begin{tabular}{|c|c|}
\hline
{\footnotesize\tt 00 01 ff $\cdots$ ff 00 30 31 30 0d 06 09 60 86 48 01 65 03 04 02 01 05 00 04 20} & HASH\\
\hline
\end{tabular}

\medskip
Signature $s$ (plaintext)
\end{center}

\noindent
The length of HASH is 32 bytes because SHA256 is used.
The 19 byte sequence starting with $\tt 30$ is from
``Abstract Syntax Notation One.''
The {\tt ff} bytes are pad bytes that are added to make the total length of the signature 2048 bits (256 bytes).
Hence there are $256-3-19-32=202$ pad bytes.

\bigskip
\noindent
Note that the above signature is the unencrypted value of $s$.
In the actual certificate $P$, $s$ is encrypted using the secret key associated with $k$.
After encryption, $s$ is still 256 bytes long.

\begin{center}
\begin{tabular}{|c|}
\hline
{\footnotesize\tt
76 b6 97 82 0f 06 b7 48 59 02 a0 2c f4 $\cdots$
fe c3 61 25 5b 1c da 77 9a a1 63 d4 49 cd}\\
\hline
\end{tabular}

\medskip
Signature $s$ (encrypted)
\end{center}

\noindent
To prove that $P$ is signed by $Q$, $s$ is decrypted using $Q$'s public key $k$.
Then if HASH matches a digest of $P$, the signing of $P$ by $Q$ is proven.

\bigskip
\noindent
Proving ``$P$ signed by $Q$'' proves that $s$ was encrypted using the secret key associated with $k$.
Only the owner of $Q$ knows the secret key.
No one can change the contents of $P$ without breaking the hash digest in $s$,
and no one can change $s$ without knowing the secret key.
Hence we can trust the contents of $P$ if we trust $Q$.

\newpage
\section{RSA}

\subsection{Public key format}

\noindent
The two integers are the modulus followed by the exponent.
This example is for RSA 2048 which has a 256 byte modulus.

\begin{verbatim}
171 290:     SEQUENCE {
175  13:       SEQUENCE {
177   9:         OBJECT IDENTIFIER rsaEncryption (1 2 840 113549 1 1 1)
188   0:         NULL
       :         }
190 271:       BIT STRING, encapsulates {
195 266:         SEQUENCE {
199 257:           INTEGER
       :             00 E0 B4 48 A2 63 21 EC 0F 9F 80 78 3B B7 27 DE
       :             8E 0A EB C7 F0 F2 41 02 D5 30 AD 61 E0 23 25 03
       :             9D 25 14 18 A2 F2 53 50 F4 2C 32 DF 0F 8D EA 90
       :             83 5B 3E 7C 64 2B 58 0E 02 E6 1F D1 DB 63 86 31
       :             F2 C2 C5 E6 2C E6 F5 89 63 3A D1 9F A9 4E 2C BE
       :             BB 03 1A 63 49 D7 5C B4 48 49 90 CB 37 3F 87 19
       :             C4 DE BC DC 3D E4 1F 69 3D EA 5A E9 88 76 B3 81
       :             69 97 E2 C6 D2 F3 F0 55 02 50 26 8F 78 C3 71 BF
       :                     [ Another 129 bytes skipped ]
460   3:           INTEGER 65537
       :           }
       :         }
       :       }
\end{verbatim}

\subsection{Signature format}

\noindent
The signature length is the same as the RSA modulus.

\begin{verbatim}
515  13:   SEQUENCE {
517   9:     OBJECT IDENTIFIER sha256WithRSAEncryption (1 2 840 113549 1 1 11)
528   0:     NULL
       :     }
530 257:   BIT STRING
       :     76 B6 97 82 0F 06 B7 48 59 02 A0 2C F4 2E 63 B2
       :     0A BD D3 8B B3 B2 0B 0C A8 35 A7 66 64 97 86 36
       :     F6 AA 0F B5 A8 AE E1 07 16 9F 72 11 60 9F DB 6E
       :     A4 60 33 D7 D9 ED 07 07 88 9A BB 4F 31 A9 EC F3
       :     42 AA 23 B3 89 EC 31 BB 8D B1 8A A5 A9 51 7B E3
       :     E6 E6 3F 6B 26 A2 B3 F2 86 6A A7 41 54 3B CC B2
       :     FA 76 8C 10 93 63 25 86 58 29 C8 7E 1D 33 C9 50
       :     FE 1D 73 78 6B 58 55 15 58 78 EC 9D A3 B9 40 A3
       :             [ Another 128 bytes skipped ]
\end{verbatim}

\newpage
\section{prime256v1}

\subsection{Public key format}

\noindent
Bit string is $04\Vert X\Vert Y$ where both $X$ and $Y$ are 32 bytes long.

\begin{verbatim}
162  89:     SEQUENCE {
164  19:       SEQUENCE {
166   7:         OBJECT IDENTIFIER ecPublicKey (1 2 840 10045 2 1)
175   8:         OBJECT IDENTIFIER prime256v1 (1 2 840 10045 3 1 7)
       :         }
185  66:       BIT STRING
       :         04 AB AA 1E 30 A0 41 00 05 C5 7F 32 E3 99 B8 BE
       :         3B 8A C1 4A A2 A3 4C CB 3C 44 97 04 4D D2 99 F1
       :         E9 CD FE 63 B3 C4 B7 05 99 1B 94 1B 87 3B 47 BA
       :         3A 76 AA 37 96 2F 89 47 31 53 EB 77 E6 43 17 D2
       :         3B
       :       }
\end{verbatim}

\subsection{Signature format}

\noindent
The two integers are $R$ followed by $S$, both 32 bytes long.
Integer format requires the first octet to be less than 128
hence in this example $R$ is 33 bytes long with a leading 00.

\begin{verbatim}
253  10:   SEQUENCE {
255   8:     OBJECT IDENTIFIER ecdsaWithSHA256 (1 2 840 10045 4 3 2)
       :     }
265  72:   BIT STRING, encapsulates {
268  69:     SEQUENCE {
270  33:       INTEGER
       :         00 9F 3F A9 AE 97 A0 48 52 AA AA AF 3E CA BA 62
       :         5F 6C 2C 46 BB 29 D0 19 A6 14 EA C0 5D 0E B9 B8
       :         D5
305  32:       INTEGER
       :         16 76 0A FC 4D 9C 6F 65 BD D2 8B CA EF C5 6E 07
       :         76 46 13 1D CF 39 A8 E3 80 D8 BD 2E B2 F9 89 0D
       :       }
       :     }
\end{verbatim}

\newpage
\section{secp384r1}

\subsection{Public key format}

\noindent
Bit string is $04\Vert X\Vert Y$ where both $X$ and $Y$ are 48 bytes long.

\begin{verbatim}
163 118:     SEQUENCE {
165  16:       SEQUENCE {
167   7:         OBJECT IDENTIFIER ecPublicKey (1 2 840 10045 2 1)
176   5:         OBJECT IDENTIFIER secp384r1 (1 3 132 0 34)
       :         }
183  98:       BIT STRING
       :         04 42 62 D9 F6 76 24 10 AE 1B 60 1F 59 45 C5 7D
       :         69 89 A3 A7 29 92 40 E6 BF FD F0 D0 20 55 BD 97
       :         5E 2B D8 BB 14 56 30 08 6E F0 02 A8 DB F4 DD C5
       :         BD DD 69 AB 39 B9 32 FC 55 D4 D5 8C 70 8E 27 3C
       :         AA A0 72 67 22 AB 1D DF 41 B5 D4 99 6D 32 7C 06
       :         ED 48 9F 31 E5 BD 10 AA 09 E7 5B 19 B6 8D 23 43
       :         27
       :       }
\end{verbatim}

\subsection{Signature format}

\noindent
The two integers are $R$ followed by $S$, both 48 bytes long.
Integer format requires the first octet to be less than 128
hence in this example $S$ is 47 bytes long with a leading 00.

\begin{verbatim}
283  10:   SEQUENCE {
285   8:     OBJECT IDENTIFIER ecdsaWithSHA256 (1 2 840 10045 4 3 2)
       :     }
295 104:   BIT STRING, encapsulates {
298 101:     SEQUENCE {
300  48:       INTEGER
       :         33 30 98 0F AA 4C 83 A1 0C 17 F9 3F 2F 05 F7 92
       :         2B 97 E9 2E E5 63 33 26 29 36 10 4F 65 2F E4 BA
       :         FF 14 09 0E 6B 07 BC 3D 8C 62 E0 4E 9C 4E B4 37
350  49:       INTEGER
       :         00 F6 6E EC 4F F9 5B 37 DC 8D E3 E9 B3 CA 13 0C
       :         5A BE F4 72 E4 4B 7A B4 BF C7 05 F1 71 83 77 68
       :         DF CF F3 CA B2 3E C5 8F E6 7E 34 B7 B4 AB 6F D5
       :         4F
       :       }
       :     }
\end{verbatim}

\end{document}
